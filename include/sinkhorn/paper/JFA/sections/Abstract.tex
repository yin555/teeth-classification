% !TEX root = ../DynamicToStatic.tex

\begin{abstract}
This article presents a new class of ``optimal transportation''-like distances between arbitrary positive Radon measures. These distances are defined by two equivalent alternative formulations: (i) a ``fluid dynamic'' formulation defining the distance as a geodesic distance over the space of measures (ii) a static ``Kantorovich'' formulation where the distance is the minimum of an optimization program over pairs of couplings describing the transfer (transport, creation and destruction) of mass between two measures. Both formulations are convex optimization problems, and the ability to switch from one to the other depending on the targeted application is a crucial property of our models. 
%
Of particular interest is the Wasserstein-Fisher-Rao metric recently introduced independently by~\cite{ChizatOTFR2015,new2015kondratyev}. Defined initially through a dynamic formulation, it belongs to this class of metrics and hence automatically benefits from a static Kantorovich formulation. Switching from the initial Eulerian expression of this metric to a Lagrangian point of view provides the generalization of Otto's Riemannian submersion to this new setting, where the group of diffeomorphisms is replaced by a semi-direct product of groups.
%
This Riemannian submersion enables a formal computation of the sectional curvature of the space of densities and the formulation of an equivalent Monge problem. 
% Then, we develop a new Kantorovich formulation and prove a correspondence result between the dynamical formulations and Kantorovich formulations. As an application, we obtain the Kantorovich formulation of the model introduced in~\cite{ChizatOTFR2015} and we develop a Gamma-convergence result of the model to  standard optimal transport.
\end{abstract}
