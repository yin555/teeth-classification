% !TEX root = ../DynamicToStatic.tex

\section*{Conclusion and Perspectives}

In this paper, we presented a unified treatment of the unbalanced transport that allows for both statical and dynamical formulations. Our key findings are (i) a Riemannian submersion from a semi-direct product of groups with an $L^2$ metric to the $\WF$ metric, which leads to the computation of the sectional curvature and a Monge formulation, (ii) a new class of static optimal transport formulations involving semi-couplings, (iii) an equivalence between these static formulations and a class of dynamic formulations. Each of these contributions is of independent interest, but the synergy between the static, the dynamic and the Monge problems allows to get a clear picture of the unbalanced transportation problem. Beside these theoretical advances, we believe that a key aspect of this work is that the proposed static formulation opens the door to a new class of numerical solvers for unbalanced optimal transport. These solvers should leverage the specific structure of the cost $c$ considered for each application, a striking example being the $\WF$ cost~\eqref{eq:GeneralizedCost:OTFR}.

% we studied the geometry of a generalized optimal transport and we established a correspondence between dynamic formulations of these models and a new Kantorovich formulation. This is a key improvement for numerical efficiency.

%%%%
\section*{Acknowledgements}

The work of Bernhard Schmitzer has been supported by the Fondation Sciences Math\'ematiques de Paris. 
%
The work of Gabriel Peyr\'e has been supported by the European Research Council (ERC project SIGMA-Vision). 
%
The work of Fran\c{c}ois-Xavier Vialard has been supported by the CNRS (D\'efi Imag'in de la Mission pour l'Interdisciplinarit\'e, project CAVALIERI).
%
\\
We would like to thank Yann Brenier for stimulating discussions as well as Peter Michor, in particular for several important references.

